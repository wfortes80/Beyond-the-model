
\documentclass[12pt]{article}
\usepackage{amsmath, amssymb}
\usepackage{geometry}
\geometry{margin=1in}
\title{Field Coupling in Emergent Geometry:\\ Neutrino Mass and Photon Behavior under $\tau_2$/$\tau_3$ Curvature Dynamics}
\author{Wayne Fortes}
\date{June 2025}

\begin{document}

\maketitle

\section*{Abstract}
This paper extends the Residual Curvature as Emergent Memory (RCEM) framework by introducing explicit coupling mechanisms between Standard Model particles and the coherence/emergence fields $\tau_2$ and $\tau_3$. We define Lagrangian-level interactions for neutrinos and photons within RCEM's six-dimensional geometric model, showing how coherence memory can modulate effective mass, propagation speed, and phase behavior. These additions open new testable predictions and bridge the model further into particle phenomenology.

 

\section*{1. Background: The RCEM Framework}
The RCEM model operates on a six-dimensional spacetime geometry:
\[
\mathcal{M}_6 = \mathbb{R}^3 \times \mathbb{R}^1 \times S^2
\]
Where:
\begin{itemize}
  \item $\mathbb{R}^3$: Observable space
  \item $\mathbb{R}^1$: Coordinate time ($\tau_1 \equiv t$)
  \item $S^2$: Compactified curvature-memory geometry
\end{itemize}

Key fields:
\begin{itemize}
  \item $\tau_2(x, \theta, \phi, t)$: Internal coherence field
  \item $\tau_3(x)$: Emergence field (activates when $E_{\tau_2} > P_{\text{Planck}}$)
  \item $\kappa(t)$: Readiness function triggering transitions
\end{itemize}

\section*{2. Neutrino Coupling}
To couple curvature-memory to neutrinos, we introduce a Yukawa-type term where the coupling constant depends on the local coherence field:
\[
\mathcal{L}_\nu = i \, \bar{\nu} \, \gamma^\mu \partial_\mu \nu - y(\tau_2) \, \bar{\nu}_L H \nu_R + \text{h.c.}
\]
Where:
\begin{itemize}
  \item $y(\tau_2) = y_0 (1 + \lambda \tau_2^2)$ modulates effective mass
  \item h.c. ensures a real-valued Lagrangian
\end{itemize}

$\tau_2$'s spatial/temporal variation creates the possibility of neutrino mass variation in high-coherence zones, which may explain oscillation anomalies or dark sector mixing.

\section*{3. Photon Coupling}
For photons, we propose a modified electromagnetic Lagrangian:
\[
\mathcal{L}_\gamma = -\frac{1}{4} (1 + \alpha \tau_2^2) F_{\mu\nu} F^{\mu\nu}
\]

This introduces an effective, curvature-modulated refractive index. Predictions include:
\begin{itemize}
  \item Phase shifts in light propagation
  \item Birefringence-like effects in curved voids
  \item Local time dilation and redshift variations
\end{itemize}

\section*{4. Emergence-Gated Coupling}
The emergence field $\tau_3(x)$ can act as an interaction envelope:
\[
\mathcal{L}_{\text{eff}} = \tau_3(x) \cdot (\mathcal{L}_\nu + \mathcal{L}_\gamma)
\]

Only in active emergence zones do these particle-level effects become significant, maintaining GR compatibility in standard environments.

\section*{5. Implications and Future Work}
This coupling framework enhances RCEM's explanatory power, linking cosmological geometry with particle-scale effects.

\textbf{Next steps:}
\begin{itemize}
  \item Simulate phase shifts near compact $\tau_3$ zones
  \item Extend to fermionic generations and charged leptons
  \item Investigate $\tau_2$ influence on polarization rotation and light cone shifts
  \item Derive constraints using neutrino observatories and radio wave dispersion data
\end{itemize}

\section*{References}
\begin{itemize}
  \item[1.] Fortes, W. (2025). \textit{Curvature Resonance and the $\tau_3$ Field}. Zenodo. DOI: 10.5281/zenodo.15598257
  \item[2.] Fortes, W. (2025). \textit{Curvature Coupling and the $\kappa_4(x)$ Function}. Zenodo. DOI: 10.5281/zenodo.15660713
\end{itemize}

\end{document}
