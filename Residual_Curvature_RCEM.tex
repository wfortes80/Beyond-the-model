
\documentclass[12pt]{article}
\usepackage{amsmath, amssymb}
\usepackage{geometry}
\usepackage{hyperref}
\usepackage{graphicx}
\usepackage{titlesec}
\geometry{margin=1in}
\titleformat{\section}{\normalfont\Large\bfseries}{\thesection}{1em}{}

\title{Residual Curvature as Emergent Memory:\\ A Field-Based Resolution of Dark Matter Phenomena}
\author{Wayne Fortes}
\date{}

\begin{document}
\maketitle

\section*{1. Abstract}
This paper redefines the concept of residual curvature using a six-dimensional curvature-memory framework incorporating a compactified 2-sphere ($S^2$) geometry and the emergence field $\tau_3$. By holding coordinate time ($\tau_1 \equiv t$) as standard and fully compatible with general relativity, we present $\kappa(t)$ as a curvature activation function that governs the onset and persistence of the $\tau_3$ field. A formal memory operator is introduced to map spacetime curvature into the internal geometry. We demonstrate that $\tau_3$ can account for observed galactic-scale gravitational phenomena typically attributed to dark matter without invoking exotic particles.

\section*{2. Framework Geometry and Field Definitions}
We model spacetime as:
\[
\mathcal{M}_6 = \mathbb{R}^3 \times \mathbb{R}^1 \times S^2
\]
Where:
\begin{itemize}
    \item $\mathbb{R}^3$ represents observable space (x, y, z)
    \item $\mathbb{R}^1$ corresponds to coordinate time ($\tau_1 \equiv t$)
    \item $S^2$ is a compactified internal curvature space encoding field memory
\end{itemize}
Field Structures:
\begin{itemize}
    \item $\kappa(t)$: Curvature readiness function, triggering field transitions
    \item $\tau_2(x, \theta, \phi, t)$: Coherence field on $S^2$, defined dynamically
    \item $\tau_3(x)$: Emergence field; activated when coherence energy exceeds Planck-threshold
\end{itemize}

\section*{3. Curvature-Memory Mapping}
We introduce a curvature-memory operator $\mathcal{C}_S(x)$ mapping spacetime curvature history into internal field states:
\[
\mathcal{C}_S(x) = \int_{t_0}^{t} \int_{\Sigma} f(R_{\mu\nu}(x', t')) W(x, x') \, d^3x' \, dt'
\]
This expression defines how historical curvature is projected into $S^2$ coherence field $\tau_2$, allowing memory encoding within compactified geometry. $W(x, x')$ is a weighting kernel (e.g., Gaussian falloff), and $f(R_{\mu\nu})$ is a curvature-based filter.

\section*{4. Emergence Dynamics and $\tau_3$ Activation}
The local coherence energy $E_{\tau_2}(x)$, integrated over $S^2$, triggers emergence when it surpasses a critical threshold, now defined as the Planck power limit $P_\text{Planck} = c^5/G$:
\[
\tau_3(x) = \frac{1}{1 + \exp[-\beta(E_{\tau_2}(x) - P_\text{Planck})]}
\]
This models $\tau_3$ as a continuous phase-transition field, activating resonance-based emergence via compactified memory \cite{fortes2025a,fortes2025b}.

\section*{5. Dark Matter Phenomena as $\tau_3$ Residual Geometry}
\begin{itemize}
    \item \textbf{Galaxy Rotation Curves:} Emergent $\tau_3$ geometry sustains curvature without baryonic mass \cite{rubin1970,sofue2001}.
    \item \textbf{Bullet Cluster:} $\tau_3$ persists spatially decoupled from hot gas via curvature-memory inertia \cite{clowe2006}.
    \item \textbf{Weak Lensing in Voids:} Fossil $\tau_3$ fields from early structure encode lensing geometry \cite{massey2007}.
\end{itemize}

\section*{6. Coupling and Observational Predictions}
\begin{itemize}
    \item $\tau_3$ modifies Einstein field equations locally via $\kappa_4(x) = \kappa_6 / (4\pi e^2 \sigma(x))$ \cite{fortes2025b}
    \item RCEM predicts spatial variation of $\tau_3$ correlated to galactic formation history
    \item Curvature without mass (e.g., void lensing) offers a falsifiable signature
\end{itemize}

\section*{7. Conclusion}
The Residual Curvature as Emergent Memory (RCEM) model, grounded within the Coherence-Driven Curvature Model (CDCM) framework, offers a fully GR-compatible explanation for dark matter phenomena via a geometric memory mechanism encoded in a compactified 2-sphere. Unlike particle-based models, RCEM reframes dark matter as a curvature echo—emergent from historical gravitational dynamics, and sustained through a resonance-based scalar field $\tau_3$.

Compared to broader emergent spacetime proposals such as the Topological Tension Residual Curvature Model (TTRCM), RCEM is distinguished by its mechanistic specificity. It features:
\begin{itemize}
    \item A dynamically stabilized extra-dimensional geometry \cite{fortes2025b}
    \item A formally defined curvature-memory operator $\mathcal{C}_S(x)$ \cite{fortes2025a}
    \item An energy threshold for emergence explicitly linked to the Planck power \cite{fortes2025b}
    \item Local causality-preserving dynamics of $\tau_3$ evolution \cite{fortes2025b}
\end{itemize}
This positions RCEM as a model with strong internal completeness, offering both explanatory clarity and testable predictions.

\begin{thebibliography}{9}
\bibitem{fortes2025a}
Fortes, W. (2025). \textit{Curvature Resonance and the $\tau_3$ Field: A Six-Dimensional Framework for Emergent Geometry}. Zenodo. \url{https://doi.org/10.5281/zenodo.15598257}

\bibitem{fortes2025b}
Fortes, W. (2025). \textit{Curvature Coupling and the $\kappa_4(x)$ Function: A Dynamic Bridge Between Extra-Dimensional Geometry and General Relativity}. Zenodo. \url{https://doi.org/10.5281/zenodo.15660713}

\bibitem{rubin1970}
Rubin, V. C., Ford, W. K. (1970). Rotation of the Andromeda Nebula from a Spectroscopic Survey of Emission Regions. \textit{Astrophysical Journal}, 159, 379.

\bibitem{sofue2001}
Sofue, Y., Rubin, V. (2001). Rotation Curves of Spiral Galaxies. \textit{Annual Review of Astronomy and Astrophysics}, 39, 137–174.

\bibitem{clowe2006}
Clowe, D. et al. (2006). A Direct Empirical Proof of the Existence of Dark Matter. \textit{Astrophysical Journal Letters}, 648(2), L109–L113.

\bibitem{massey2007}
Massey, R. et al. (2007). Dark matter maps reveal cosmic scaffolding. \textit{Nature}, 445, 286–290.
\end{thebibliography}

\end{document}
